\chapter*{Conclusion}
\addcontentsline{toc}{chapter}{Conclusion}

It is clear from the above discourse that accounting for electronic polarization is crucial for an accurate response of the lipid electrometer in simulation. 


In conclusion, the results suggest that calcium ions bind specifically to the phosphate oxygens, occasionally interacting also with the carbonyls of the PC lipids. This is in a qualitative agreement with previous conclusions from several experimental studies~\citep{hauser76, hauser78, herbette84, cevc90, binder02}. However, the present results suggest, in 
agreement with experiments, an overally weaker binding to the bilayer, in particular with a lower relative binding affinity to the carbonyls than inferred from previous MD simulation studies~\citep{bockmann03, bockmann04, melcrova16, javanainen17}. 

In summary, the results from ECC-lipids suggest 
that the exchange of calcium between the POPC bilayer and the solvent 
occurs in the order of $\sim$10--100~ns, 
which is significantly faster than observed in simulations 
with other presently available non-polarizble models of lipids~\citep{javanainen17, catte16}. 
Our results suggest that simulations with a characterisitc length of several hundred nanoseconds 
are necessary and yet sufficient for 
simulating binding of alkali and alkali earth ions to phospholipid bilayers 
in equilibrium when more realistic \emph{polarizable} force fields are used. 
