\chapter*{Conclusion}
\addcontentsline{toc}{chapter}{Conclusion}

%==============================
You only summarize the technical aspects of your results. 
But what is the broader impact and where is it leading?
Plus you do not even mention the transmembrane potential.
%==============================


We performed simulations of phospholipid bilayers at various salt concentrations. 
We used the most abundant representants of phospholipids
to form neutral (POPC) or negatively charged (5\,POPC:1\,POPS) bilayers, 
for which we have provided a detailed insight 
into the interactions with 
\ce{Na^+}, \ce{K^+}, and \ce{Ca^{2+}} cations. 

The affinity of the cations 
was measured using the concept of a lipid electrometer 
introduced by \citet{seelig87} 
and described in section~\ref{section:electrometer}.
The head group order parameters $\alpha$ and $\beta$ in POPC
%(see Fig.~\ref{fig:simVSexpNOions} for the definition of the order parameters)
are experimentally observed to change
proportionally to the bound charge per lipid. 
Such changes can then be related to the amount of bound ions.
In our publication \citep{catte16},
we have shown that such order parameters can be accurately determined also from MD simulations
and their changes correlate with the amount of bound charge in PC bilayers. 
However, all models of PC bilayers examined in that work
overestimate the binding affinity of \ce{Ca^{2+}}, 
and many of them also that of \ce{Na^+}. 

A major improvement over currently available non-polarizable force fields
was achieved by developing new models of phospholipids, 
ECC-lipids,
which are described in the section~\ref{section:ecc-lipids}
and in the publication \citep{melcr18}. 
ECC-lipids account for electronic polarization
using the electronic continuum correction,  
an implicit model of electronic polarizability, 
which was introduced in section~\ref{section:ecc}. 
The performed simulations with ECC-lipids 
provide a proof of concept of the applicability of this approach 
to both neutral and charged molecules. 

Our simulations with ECC-lipids suggest that
\ce{Na^+} and \ce{Ca^{2+}} cations
interact specifically with the phosphate and carboxylate groups of PC and PS, 
and occasionally also with the carbonyl groups. 
\ce{K^+} interacts only very weakly with the bilayers
affecting slightly only carboxylate groups in POPS,
which are more exposed to the solvent compared to phosphate groups. 
An overall weaker binding of cations to phospholipid bilayers is observed 
compared to previous MD simulation studies~\citep{nmrlipids_proj4, catte16, bockmann03, bockmann04, melcrova16, javanainen17}. 
Importantly,
our simulations show for the first time a \emph{quantitative} agreement with the lipid electrometer concept
for POPC and also for POPS with all the studied cations. 
For instance, the small differences 
in the responses of the order parameter $\beta$ in POPS
between the 
\ce{Na^+}, \ce{K^+}, and \ce{Ca^{2+}} cations
are captured well by our model. 
Also, the exchange of calcium between a phospholipid bilayer and solvent 
occurs at the order of 10--100~ns, 
which is in accord with experiments and also 
significantly faster than the time scales from simulations 
with other presently available non-polarizable models of lipids~\citep{melcrova16, javanainen17, catte16}. 
Nevertheless, even with ECC-lipids,
the reversible process of calcium binding to phospholipid bilayers in equilibrium
requires simulations of a characterisitc length of several hundreds of nanoseconds 
for the neutral bilayers,
while almost an order of magnitude longer lengths 
are required for the negatively charged bilayers. 


