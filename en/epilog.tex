\chapter*{Conclusion}
\addcontentsline{toc}{chapter}{Conclusion}

Motivated by cellular processes, 
which involve ions and cell membranes as major actors, 
we have investigated model phospholipid membranes and their interactions 
with biologically relevant ions. 
The controlled concentrations of \ce{Na^+} and \ce{K^+} cations 
on either side of the membrane of neurons
forms the transmembrane potential. 
Neurons vary the concentrations of theses cations
to modulate the transmembrane potential 
and to conduct electrical signals along the axons. \citep{Knudsen_book2002, Storace2015, Sung2015}
In our work \citep{melcr16},
we compared two methods for modeling of the transmembrane potential in molecular simulations, 
i.e., 
the constant electric field method \cite{Roux1997,Roux2008, gumbart_constant_2012}, 
and the ion imbalance method \cite{sachs04_potential, kutzner_computational_2011}. 
We have proven the two methodologies to be equivalent, 
at least for electrolytes formed by the same monovalent ions on both sides of the membrane. 
%This was demonstrated by simulations, 
%in which we simultaneously applied both methods
%with an opposite polarity. 
%The results from such a setup were indistinguishable 
%from simulations without voltage within the achievable accuracy. 
While
the structure of the bilayer remains almost unchanged by the transmembrane potential, 
the induced electric field in the hydrophobic core of the bilayer
affects the orientation of water molecules at its interface 
highlighting its importance in electroporation. \citep{bu2017mechanics}


Next,
we performed extensive sets of simulations of phospholipid bilayers
with different salts at varying concentrations
to study the binding of cations to model membranes. 
We used the most abundant representants of phospholipids
to form neutral (POPC) or negatively charged (5\,POPC:1\,POPS) bilayers, 
for which we have provided a detailed insight 
into the interactions with 
\ce{Na^+}, \ce{K^+}, and \ce{Ca^{2+}} cations. 
The affinity of the cations 
was measured using the concept of a lipid electrometer 
introduced by \citet{seelig87} 
and described in section~\ref{section:electrometer}.
The head group order parameters $\alpha$ and $\beta$ in POPC
%(see Fig.~\ref{fig:simVSexpNOions} for the definition of the order parameters)
are experimentally observed to change
proportionally to the bound charge per lipid. 
Such changes can then be related to the amount of bound ions.

In our publications \citep{catte16, nmrlipids_proj4},
we have shown that such order parameters can be accurately determined also from MD simulations
and their changes correlate with the amount of bound charge in phospholipid bilayers, 
despite the inaccuracies in their actual structures without salts \citep{botan15}. 
It was found, however, that
none of the force fields examined in those works 
provided a sufficient accuracy for interpreting 
the experimentally measured structural changes induced by salt concentrations
and cation-lipid stoichiometries. 
While there were several models
that predicted realistic binding affinities of \ce{Na^+} to PC bilayers,
all existing models overestimated the binding affinity of \ce{Ca^{2+}}
unless ad hoc specific repulsive potentials between the cations and the bilayer were applied. \citep{catte16, nmrlipids_proj4}
Thus,
we identified the strong binding of \ce{Na^+} and \ce{Ca^{2+}} cations
in existing simulation models as a computational artifact.  
Such excessive amounts of cations 
would form effectively positively charged membranes
even at physiological concentrations
affecting interactions with any charged molecules. 
For instance,
the total charge of proteins
in prokaryotes and also eukaryotes 
is mostly negative,
\citep{link1997identifying, link1997comparing, urquhart1998comparison, schwartz2001whole, knight2004global}
Thus,
positively charged membranes would promote non-specific adsorption of proteins on their surface 
contrary to experiment.  
\citep{junkova2016, lingwood2010lipid, sekerevs2015song} 


A major improvement over currently available non-polarizable force fields
was achieved by developing new models of phospholipids, 
the so called ECC-lipids,
which are described in section~\ref{section:ecc-lipids}
and in the publication \citep{melcr18}. 
In contrast to the models studied in the works \citep{catte16, nmrlipids_proj4},
ECC-lipids account for electronic polarization
via the electronic continuum correction,  
which was introduced in section~\ref{section:ecc}. 
In short, 
ECC is an implicit model of electronic polarizability,
which can be straightforwardly implemented into current force fields 
by scaling charges.  
In section~\ref{section:ecc-lipids},
we demonstrated on the cases of PC, PS, and PE phospholipids
that it is sufficient to also scale the Lennard-Jones parameters $\sigma$ of the affected atoms
to reach a structural agreement with x-ray scattering experiments. 
Our new model, ECC-lipids,
provides a proof of concept of the applicability of ECC
to charged as well as neutral molecules. 



Our simulations with ECC-lipids suggest that
\ce{Na^+} and \ce{Ca^{2+}} cations
interact specifically with the phosphate and carboxylate groups of PC and PS 
and occasionally also with the carbonyl groups. 
\ce{K^+} interacts only very weakly with the bilayers
affecting slightly only carboxylate groups in POPS,
which are more exposed to the solvent compared to phosphate groups. 
In overall, weaker binding of cations to phospholipid bilayers is observed 
compared to previous MD simulation studies~\citep{nmrlipids_proj4, catte16, bockmann03, bockmann04, melcrova16, javanainen17}. 
Importantly,
our simulations show for the first time a \emph{quantitative} agreement with the experimental lipid electrometer concept
for POPC and also for POPS with all the studied cations. 
For instance, the small differences 
in the responses of the order parameter $\beta$ in POPS
between the 
\ce{Na^+}, \ce{K^+}, and \ce{Ca^{2+}} cations
are captured well by our model. 
Also, the exchange of calcium between a phospholipid bilayer and solvent 
occurs at the order of 10--100~ns, 
which is in accord with experiments and also 
significantly faster than the time scales from simulations 
with other presently available non-polarizable models of lipids~\citep{melcrova16, javanainen17, catte16}. 
Nevertheless, even with ECC-lipids,
the reversible process of calcium binding to phospholipid bilayers in equilibrium
requires simulations of a characteristic length of several hundreds of nanoseconds 
for the neutral bilayers,
while almost an order of magnitude longer lengths 
are required for the negatively charged bilayers. 
In summary,
our results are in accordance with the works,
which suggest that monovalent cations (with the exception of \ce{Li^+}) 
exhibit negligible binding to phospholipid bilayers, 
while multivalent cations interact significantly 
\citep{cevc90,tocanne90, hauser76,hauser78,herbette84,altenbach84,clarke99,binder02,pabst07,uhrikova08,filippov09}.


Treatment  of the electronic polarization 
was shown to have a dramatically positive impact on the accuracy of the description of interactions
between phospholipids and cations. 
The presented application of ECC to lipids
constitutes a pivotal work for its future adaptations 
to also other compounds, especially charged and zwitterionic, 
e.g., proteins or nucleic acids,
for which we expect improvements in computational description in a similar range. 

