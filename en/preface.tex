\chapter*{Introduction}
\addcontentsline{toc}{chapter}{Introduction}

 \textbf{Relevance and significance.} \\
 Can be quite short. \\
 cytosol and cytoplasm. Electrolytes in biology, salts, Important cations (anions). 
 K, Na, Ca -- and -- Li, Mg, Zn.
 List all of them, but concentrate only on K, Na and Ca (Cl), as they are the most abundant and they take part in neural singalling and transmembrane potential gradients (Neurons, hear-beat, hearing)
 I can probably take hints from our recent publication on ECC-Ca by \citet{martinek17}. 

\todo{dissect the following sections and paragraphs}
\section{Transmembrane voltage, neurons, action potential}

 Unequal distribution of cations forms transmembrane potential on cell membranes -- neurons, synaptic bouton (very briefly, it's a motivation here). 

All eukaryotic cells maintain a non-zero transmembrane potential across their plasma membranes. 
The voltage difference from the exterior of the cell in a resting state to the interior is usually in the range from -10 to -100~mV. \cite{MolBiolCell, sten-knudsen_biological_2002} 
Its magnitude is determined by the permeability of the membrane to specific ionic species via ion channels, by the intra- and extra-cellular ionic distributions, as well as by active ion transport across the membrane  \cite{sten-knudsen_biological_2002}. 
In neurons, changes in the membrane potential enable these cells to conduct electrical signals along their axons.  
In the resting state of the neuron, the transmembrane voltage has a value around -70~mV \cite{sten-knudsen_biological_2002}. 
During neural activity, however, the transmembrane potential undergoes a dramatic change. 
This process happens in the range of milliseconds, and it is mainly driven by the action of sodium and potassium channels. 
The openings and closings of these channels induce non-zero fluxes of ions across the membrane. 
This leads to a sudden polarity inversion of the transmembrane voltage, which then propagates along the neural axon. 
This avalanche process of membrane depolarization is called an action potential. \cite{sten-knudsen_biological_2002}


In physiology, an action potential is a short-lasting event in which the electrical membrane potential of a cell rapidly rises and falls with a particular profile. 
Action potentials occur in several types of animal cells, called excitable cells, which include neurons, muscle cells, and endocrine cells, as well as in some plant cells. 
In neurons, they play a central role in cell-to-cell communication. 
In other types of cells, their main function is to activate intracellular processes. 
In muscle cells, for example, an action potential is the first step in the chain of events leading to contraction. 
In beta cells of the pancreas, they provoke release of insulin. 
\textbf{Action potentials in neurons} are also known as \textbf{"nerve impulses" or "spikes"}, and the temporal sequence of action potentials generated by a neuron is called its \textbf{"spike train"}. 
A neuron that emits an action potential is often said to \textbf{"fire"}.


   Action potentials are generated by special types of \textbf{voltage-gated ion channels} embedded in a cell's plasma membrane. These channels are shut when the membrane potential is near the resting potential of the cell, but they rapidly begin to open if the membrane potential increases to a precisely defined threshold value. When the channels open (in response to depolarization in transmembrane voltage), they allow an inward flow of sodium ions, which changes the electrochemical gradient, which in turn produces a further rise in the membrane potential.
   
    The rapid influx of sodium ions causes the polarity of the plasma membrane to reverse, and the ion channels then rapidly inactivate. As the sodium channels close, sodium ions can no longer enter the neuron, and then they are actively transported back out of the plasma membrane. Potassium channels are then activated, and there is an \textbf{outward current of potassium ions}, returning the electrochemical gradient to the resting state. After an action potential has occurred, there is a transient negative shift, called the \textbf{afterhyperpolarization} or \textbf{refractory period}, due to additional potassium currents. This is one of the helping mechanisms that \textbf{prevents an action potential from traveling back} the way it just came.
    
    In animal cells, there are \textbf{two primary types of action potentials}. One type is generated by \textbf{voltage-gated sodium channels}, the other by\textbf{ voltage-gated calcium channels}. Sodium-based action potentials usually last for under one millisecond, whereas calcium-based action potentials may last for 100 milliseconds or longer. In some types of neurons, slow calcium spikes provide the driving force for a long burst of rapidly emitted sodium spikes. In cardiac muscle cells, on the other hand, an initial fast sodium spike provides a "primer" to provoke the rapid onset of a calcium spike, which then produces muscle contraction.


In experiments, the transmembrane potential can be monitored using microelectrodes \cite{Sakmann1976,Grinvald1987} or voltage-sensitive fluorescent probes. \cite{Han2014,Akemann2010,Perron2009a,Mutoh2009,Barnett2012,Wooltorton2012,Akemann2009,Fisher2008,Obaid2004}  
The information about the transmembrane potential obtained from the experiment is of a limited microscopic resolution, showing features like current-voltage characteristics. 
In contrast, computer simulations, such as molecular dynamics, provide an atomistic picture of the system. 
To date, there exist two methods for modeling of the transmembrane potential in molecular simulation -- the constant electric field method \cite{roux_influence_1997, tieleman_voltage-dependent_2001, roux_membrane_2008, gumbart_constant_2012, sin_asymmetric_2015} and the ion imbalance method 
\cite{sachs_changes_2004, delemotte_modeling_2008}. 
Both of these methods have been successfully used to study the electroporation phenomenon or to induce a transmembrane voltage when a voltage-sensitive protein was embedded within a lipid bilayer 
\cite{Vargas2012, bockmann_kinetics_2008, gumbart_constant_2012, kutzner_computational_2011, casciola_molecular_2014}. 

\section{Last paragraph}
 End up with some concluding statement on how membrane cation interactions are abundant and also relevant in biology.
 Signalling? (i.e. lipids, POPS, and some cations, Ca)

