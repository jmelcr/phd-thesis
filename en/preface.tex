\chapter*{Preface}
\addcontentsline{toc}{chapter}{Preface}

Cellular membranes are important and evolutionarily very old biological structures. 
\citep{MolBiolCell} 
The first primitive predecessors of cells already bear hints of membranes 
separating their inner environment from the outer world. 
Current organisms often contain a multitude of immensely complex membranes, 
each serving many functions. 
Processes in cellular membranes are thus crucial for life. 

My work is motivated by processes,
which involve interactions of biologically relevant ions with cellular membranes. 
For instance in neurons, 
the fusion of synaptic vesicles containing neurotransmitter with neuronal cell membranes 
is controlled by a divalent cation \ce{Ca^{2+}}. \citep{Berridge2003, Clapham2007}
This process is triggered by a change in 
the transmembrane potential across the neuronal plasma membrane,
which is modulated by the exchange of the monovalent cations \ce{Na^+} and \ce{K^+}. \citep{Knudsen_book2002}
The transmembrane potential in atomistic simulations can be modeled by two approaches, 
the constant electric field method, 
and the ion-imbalance method. 
The methodological differences between them raise the following questions:
Do they provide the same results? 
Can they be used interchangeably?
What happens to the membrane under voltage?

Until recently,
there was no consensus on the binding of 
\ce{Na^+}, \ce{K^+}, and \ce{Ca^{2+}} cations
to biological membranes -- 
simulations and some experiments   \citep{berkowitz12, vacha09a, harb13}
do not reproduce other experiments \citep{roux90, pabst07, akutsu81}. 
From the point of simulations,
all currently available models 
require improvements to reproduce quantitatively structures and interactions with cations. 
Here, the following questions arise:
Is the missing electronic polarization in standard non-polarizable simulations responsible for the discrepancy?
If yes, how crucial is it for the interactions of phospholipid bilayers with biologically relevant cations,
and can it be effectively accounted for by rescaling charges?
Can we obtain in this way realistic structures of phospholipid bilayers with interacting cations at atomistic resolution?


