\chapter*{Introduction}
\addcontentsline{toc}{chapter}{Introduction}

Cellular membranes are splendid biological structures. 
The first primitive predecessors of cells already bear hints of membranes 
separating their inner environment from the outer world. 
Current organisms often contain a multitude of immensly complex membranes, 
each serving many functions. 
Processes in cellular membranes are crucial for life. 

%In this thesis,
%Simulation of processes in cellular membranes,
%I focus on the processes, 
My work is mostly motivated by the processes,
which involve interactions with ions. 
Excitable cells like neurons 
rely on the exchange of the monovalent cations \ce{Na^+} and \ce{K^+}
enabling them to conduct electrical signals. 
Fusion of membranes includes interactions with the divalent cation \ce{Ca^{2+}}.
I accurately quantify the interactions of these cations, 
\ce{Na^+}, \ce{K^+} and \ce{Ca^{2+}},
with model biological membranes
using classical molecular dynamics simulation. 
In order to achive this,
I developed improved theoretical models
of phospholipids, major components of cellular membranes.
The models account for electronic polarization
using Electronic continuum correction,  
an implicit model of electronic polarizability. 
My simulations provide a proof of concept 
for the applicability of this approach 
to both neutral and charged molecules. 
I show that accounting for electronic polarization 
is necessary for accurate description of interactions 
between ions and phospholipids. 

