\chapter*{Preface}
\addcontentsline{toc}{chapter}{Preface}

Cellular membranes are important and evolutionarily very old biological structures. 
\citep{MolBiolCell, Knudsen_book2002} 
The first primitive predecessors of cells already bear hints of membranes 
separating their inner environment from the outer world. 
Current organisms often contain a multitude of immensly complex membranes, 
each serving many functions. 
\emph{Processes in cellular membranes} are crucial for life. 

%In this thesis,
%Simulation of processes in cellular membranes,
%I focus on the processes, 
My work is motivated by the processes,
which involve interactions of biologically relevant ions with membranes. 
Excitable cells like neurons 
rely on the exchange of the monovalent cations \ce{Na^+} and \ce{K^+}
enabling them to conduct electrical signals. 
Fusion of synaptic vesicles with neuronal cell membranes 
is controlled by a divalent cation \ce{Ca^{2+}}.
In this work, 
I accurately quantify the interactions of these cations, 
i.e.,
\ce{Na^+}, \ce{K^+} and \ce{Ca^{2+}},
with model biological membranes
using classical molecular dynamics simulation. 
In order to achive this goal,
I developed an improved force field
of phospholipids as the major components of cellular membranes.
These models account for electronic polarization
using the electronic continuum correction,  
which is an effective way of accounting for electronic polarization via charge rescaling. 
My simulations provide a proof of concept 
for the applicability of this approach 
to both neutral and charged lipids. 
I demonstrate that accounting for electronic polarization 
is necessary for accurate description of interactions 
between ions and phospholipids. 

%==============================
Extend with the transmembrane potential modeling work.
%==============================
