\chapter{Interactions of ions with phospholipid membranes}
\label{chap:results}

Biological membranes naturally exist in a weak electrolytic solution of \ce{KCl} on the intracellular side, and of \ce{NaCl} on the extracellular side. 
The biological relevance of these ions reaches from relatively simple osmotic effects to the complex processes in neural signalling. 

Calcium is an important cation in biology, 
which takes part in many signalling pathways, e.g. triggering of the release of neurotransmitter in neurons,
and processes such as regulating cardiac rythm in heart. 
The interaction of \ce{Ca^{2+}} with phospholipid membranes has recieved attention recently from both experiments and simulations \citep{melcrova16, javanainen17}.

In the previous chapter, we presented new classical MD models of phospholipids,
which account for electrionic polarization via ECC (introduced in section~\ref{section:ecc}). 
It was demonstrated that such models, termed ECC-lipids, 
meet current accuracy standards in the field 
by comparing them with NMR order parameters and X-ray diffraction data (section~\ref{section:ecc-lipids}. 

In this chapter, we will provide detailed insight into the interactions of these ions with neutral and negatively charged model membranes,
namely with a POPC bilayer and with a negatively charged bilayer with a composition of 5~POPC:1~POPS. 
We employ our newly developed models of ions and phospholipids, 
ECC-ions \citep{martinek17, kohagen16, Pluharova2014} and ECC-lipids \citep{melcr18}, 
which excel any state-of-the-art model of ions or lipids in terms of lipid-ion interactions. 

First, we summarize the literature knowledge on the interactions of ions with membranes in experiments and simulations.
Then we demonstrate the outstanding accuracy of the newly developed model of POPC, ECC-POPC, 
in the responses of the head group order parameters.
The simulation study with a cationic surfactant 
validates ECC-POPC as an accurate model of the lipid electrometer concept observed in NMR experimnents. 
At last, we will provide detailed insight into the binding of cations to the neutral and negatively charged bilayers. 
We put extra stress on the interactions with \ce{Ca^{2+}}, 
for which we present the first simulation results that are in \emph{quantitative} agreement with experiments. \citep{catte16, melcr18}





\section{Binding of cations to phospholipid bilayers and lipid electrometer concept from experiments and simulations}
\label{section:electrometer_exp_sim} 

The response of the head group order parameters 
to a certain amount of bound charge in the bilayer 
was calibrated using monovalently charged surfactants in \citep{scherer89}. 
After such a calibration,
binding affinities of free cations can be estimated from the measured head group order parameter changes. \citep{scherer89}
This forms the lipid electrometer concept (introduced in section\ref{section:electrometer}),
which can be used to directly compare experimental measurements with MD simulations. 
We performed such a comparison for both neutral and negatively charged membranes with \ce{NaCl} and \ce{CaCl2} 
for a large array of MD simulations produced within NMRlipids open collaboration platfrom \citep{nmrlipids}. 
It was concluded  that the binding affinities of cations are overestimated in almost all models studied in~\citep{catte16} (Fig.~\ref{fig:catte16}) and \citep{nmrlipids_proj4}. 


The small head group order parameter response of a neutral POPC bilayer to \ce{NaCl} from the experiments by~\citet{seelig87} is captured by only a few models, 
namely Lipid14 \citep{dickson14}, Orange (see supplementary infromation in \citep{catte16}) and CHARMM36 \citep{klauda10}. 
However, the experimentally measured head group order parameter response to \ce{CaCl2} is captured only \emph{qualitatively} by the models. \citep{catte16}
In addition, none of the employed models in that study reproduces the order parameters without any salt concentration
within experimental error, indicating structural inaccuracies of varying severity in all of them~\citep{botan15}.
In summary, all models of a POPC bilayer examined in \citep{catte16} 
overestimate the response of head group order parameters and/or binding affinity of \ce{CaCl2} to such bilayers. 



\begin{figure}[tbp]
  \centering
  \includegraphics[width=\figwidthfull]{../img/OrderParameterIONSchanges.pdf}
  \caption{\label{fig:catte16}
    Changes of the head group order parameters $\beta$ (top row) and $\alpha$ (bottom row) 
    to increasing concentrations of \ce{NaCl} (left column) and \ce{CaCl2} (right column)
    Results from experiments 
    (DPPC from Ref.~\citep{akutsu81}, POPC from Ref.~\citep{altenbach84}) 
    are compared with simulations gathered through NMRlipids open collaboration platform \citep{nmrlipids} in the work by \citet{catte16}. 
    Note that none of the employed models in this figure reproduces the order parameters without any salt concentration
    within experimental error, indicating structural inaccuracies of varying severity in all of them~\citep{botan15}.
  }
\end{figure}


Similar conclusions as for POPC bilayers are also found for 
negatively charged membranes with a composition 5\,PC:1\,PS in \citep{nmrlipids_proj4}. 
Although the structure of the lipids in a POPS bilayer with only \ce{Na^+} counterions
is again not captured within the experimental error bars,
the small response of the order parameters of PC and PS head groups 
to incresing concentrations of \ce{NaCl} in the mixed bilayer with a composition 5\,PC:1\,PS 
is maintained in a few models, 
namely Lipid17 \citep{lipid17-future} and less well in MacRog \citep{maciejewski14}. 
Interestingly, the response to \ce{CaCl2} is overestimated by all employed models but one, CHARMM36 obtained from \url{http://charmm-gui.org/} \citep{jo08, lee15},
which has incorporated a correction for the observed excessive binding of \ce{Ca^{2+}} to PC and PS similar to the correction for \ce{Na^+} \citep{venable13}. 
With such a model, the response of the head group order parameters of PC 
to increasing concentrations of \ce{CaCl2} in a mixed negatively charged bilayer
is significantly reduced by the employed correction for \ce{Ca^{2+}}
even below the response measured experimentally. 
Hence, it is the only model in the study,
which underestimates the response of the lipid electrometer for the negatively charged membrane
contrasting with the results from the neutral POPC bilayer in \citep{catte16}. 



\begin{figure}[tb!] 
  \centering 
  \includegraphics[width=\figwidth]{../img/ecc_popc/PN_angle_OrdPars-A-B_L14-ECCL17_q80_sig89_surf.pdf} 
  \caption{\label{OrderParameterCHANGESsurf} 
    The changes of head group order parameters and P-N vector orientation as a function of 
    a molar fraction of the cationic surfactant dihexadecyldimethylammonium in a POPC bilayer 
    from simulations and experiments \citep{scherer89} at 313 K.
  } 
\end{figure} 

 

In order to distinguish, whether the observed discrepancy of the head group order parameter changes between simulations and experiments 
arise from incorrect sensitivity of the head group or from mostly excessive binding of cations to the phospholipid bilayer,
we performed simulations of a neutral POPC bilayer with varying amounts of the cationic surfactant dihexadecyldimethylammonium, 
as was measured in the experimental work by \citet{scherer89}.
The amount of bound charge per PC 
in such systems is given simply by the molar fraction of the cationic surfactants, 
as essentially all of the surfactants locate to the lipid bilayers 
due to the two long hydrophobic tails.
The NMR measurements of such systems  
can be used to validate the sensitivity of lipid headgroup order parameters 
(i.e. the coefficient $m_i$ in Equation~\ref{OPchangeEQ}) 
to the amount of bound charge in simulations. 
The changes of the headgroup order parameters in a POPC bilayer with an increasing amount of 
the cationic surfactant from simulations and experiments~\citep{scherer89} are shown in Fig.~\ref{OrderParameterCHANGESsurf}.
In line with Equation~\ref{OPchangeEQ},
experiments and also both MD simulation models show approximately linear decrease of the head group order parameters $\alpha$ and $\beta$.

Results from simulations with two models are shown.
We employ our newly developed model ECC-POPC, introduced in the previous chapter, 
which accounts for electronic polarizability using ECC \citep{leontyev14}.
For a direct comparison, we also show results for Lipid14, a standard model by \citet{dickson14},
which served as a starting point in the development of ECC-POPC. 
This model was also used in the works \citep{catte16, nmrlipids_proj4}. 

While the slope of the response of order parameters $\alpha$ and $\beta$ 
from the simulation with ECC-POPC model 
is in a very good agreement with the experiments, 
the slope from Lipid14 is too steep
suggesting that the overestimated response observed also in \citep{catte16}
arises at least in part from an overly sensitive response of the head group. 

The increasing amount of cationic surfactant in the bilayer
also affects the P-N vector, 
which is defined as the angle between the connector of the phosphorus and nitrogen atoms and the membrane normal. 
Similarly to the order parameters, 
there is a linear dependence on the amount of the cationic surfactant
as shown in Fig.~\ref{OrderParameterCHANGESsurf}. 
Although the structure of ECC-POPC model without any ions does not agree with NMR experiments within errors,
such a model reproduces well the changes of the order parameters.
It follows that using this model,
we can write an approximate relation between the changes of the order parameters and the P-N vector mean orientation. 
For the change of the order parameter $\alpha$, $\Delta S^\alpha$, we arrive at
\begin{equation}
\Delta \vec{PN} = (186 \pm 9) \cdot \Delta S^\alpha .
\end{equation}

After reproducing the concept of a lipid electrometer with a known amount of bound surface charge,
ECC-lipids will be employed in the study of interactions with aquaeous ions, namely with \ce{K^+}, \ce{Na^+} and \ce{Ca^{2+}}. 












\section{Interactions of neutral and negatively charged phospholipid membranes with Na$^+$ and K$^+$ cations}
\label{section:lip-ion_k_na}

\begin{figure}[tbp!] 
  \centering 
  \includegraphics[height=10 cm]{../img/ecc_popc/OrdPars-A-B-PNvec_L14-ECC-lipids_NaCl.pdf}
  \includegraphics[height= 9 cm]{../img/ecc_popc/OrdPars-A-B-PNvec_L14-ECC-lipids_KCl.pdf}
  \caption{\label{fig:delta_ordPar_NaCl} 
    Changes of the head group order parameters of a POPC bilayer as a function of \ce{NaCl} (left) and \ce{KCl} (right) concentration 
    in bulk ($C_{ion}$) from simulations with different force fields at 313 K together with  
    experimental data for DPPC (323\,K) \citep{akutsu81} and POPC (313\,K) \citep{altenbach84}. 
    Simulation data with Lipid14 and Åqvist ion parameters at 298 K are taken directly from 
    Refs.~\citep{lipid14POPC0mMNaClfiles,lipid14POPC1000mMNaClfiles}. 
  } 
\end{figure} 
 
 
\begin{figure}[tbp!] 
  \centering 
  \includegraphics[width=\figwidthsmall]{../img/ecc_pops/order_parameters_changes_ecc-lip_L14_A-B-PN-COO_POPC_nacl.pdf} 
  \includegraphics[width=\figwidthsmall]{../img/ecc_pops/order_parameters_changes_ecc-lip_L14_A-B-PN-COO_POPS_nacl.pdf} 
  \caption{\label{fig:delta_ordPar_NaCl_PCPS} 
    Changes of the head group order parameters $\alpha$, $\beta$ and the orientations of the carboxylate group and the P-N vector  
    of POPC (left) and POPS (right) phospholipids in a POPC:POPS 5:1 bilayer as a function of \ce{NaCl} concentration 
    in bulk ($C_{ion}$) from simulations with different force fields at 298 K.
    Because data with \ce{NaCl} are not available for POPC, 
    we show experimental data for \ce{LiCl} (dashed line, left) \citep{roux90}
    as an upper bound for the magnitude of the response to \ce{NaCl}, 
    which has a lower affinity to phospholipid bilayers compared to \ce{LiCl}. 
    The orientation of the \ce{COO^-} group is defined as 
    the connector from the $\beta$ carbon to the carbon in \ce{COO^-} (stars, bottom right). 
  } 
\end{figure} 


\begin{figure}[tbp!] 
  \centering 
  \includegraphics[width=\figwidthsmall]{../img/ecc_pops/order_parameters_changes_ecc-lip_L14_A-B-PN-COO_POPC_kcl.pdf} 
  \includegraphics[width=\figwidthsmall]{../img/ecc_pops/order_parameters_changes_ecc-lip_L14_A-B-PN-COO_POPS_kcl.pdf} 
  \caption{\label{fig:delta_ordPar_KCl_PCPS} 
    Changes of the head group order parameters $\alpha$, $\beta$ and the orientations of the carboxylate group and the P-N vector  
    of POPC (left) and POPS (right) phospholipids in a POPC:POPS 5:1 bilayer as a function of \ce{KCl} concentration 
    in bulk ($C_{ion}$) from simulations with different force fields and experiments at 298 K. \citep{roux90}
    The orientation of the \ce{COO^-} group is defined as 
    the connector from the $\beta$ carbon to the carbon in \ce{COO^-} (stars, bottom right). 
  } 
\end{figure} 
 


It was presented in the previous section,
that accounting for electronic polarization,
a distinct feature of ECC-lipids compared to other lipid models, 
is crucial for an accurate description of the response of the POPC electrometer. 
In this section,
we employ ECC-lipids to simulate neutral and negatively charged bilayers 
in the presence of monovalent salts, namely \ce{KCl} and \ce{NaCl}. 
The results are compared using the concept of a lipid electrometer 
with NMR measurements and simulations in Figures \ref{fig:delta_ordPar_NaCl}, \ref{fig:delta_ordPar_NaCl_PCPS} and \ref{fig:delta_ordPar_KCl_PCPS}. 

Interactions of \ce{Na^+} with neutral POPC bilayer were studied in \citep{catte16}, 
and with negatively charged mixed 5\,PC:1\,PS bilayer in \citep{nmrlipids_proj4}. 
It was concluded that the interactions are generally overestimated in magnitude in almost all models 
but Lipid14 \citep{dickson14}, resp. Lipid17 \citep{lipid17-future}, 
which yields semi-quantitative agreement with the small changes of the order parameters measured experimentally
when used with the model of ions by \citet{aqvist90} (Fig.~\ref{fig:catte16}). 
However, when used with a more accurate model of ions by \citet{Pluharova2014, martinek17},
the model overestimates the binding affinity of \ce{Na^+}
measured with lipid electrometer concept in Fig.~\ref{fig:delta_ordPar_NaCl}. 
In line with the previous work \citep{catte16}, the results suggest that improvements 
in the lipid parameters are required for more accurate interactions even with monovalent cations. 

The results from simulations combining the models ECC-lipids \citep{melcr18} and ECC-ions \citep{martinek17, kohagen16, Pluharova2014} 
exhibit an improved behavior of the POPC and POPS head group order parameters as a function of \ce{NaCl} or \ce{KCl} concentrations, 
as plotted in Figs.~\ref{fig:delta_ordPar_NaCl}, \ref{fig:delta_ordPar_NaCl_PCPS} and \ref{fig:delta_ordPar_KCl_PCPS}. 

The interaction with \ce{K^+}, which binds very weakly to both neutral and negatively charged membranes, 
renders a qualitatively different response of the order parameter $S^\beta$ in POPS in the mixed negatively charged membranes. 
While the order parameter $S^\beta$ increases for both \ce{Na^+} and \ce{Ca^{2+}},
it decreases in the presence of \ce{K^+}.
This feature is \emph{qualitatively} captured by a few models in \citep{nmrlipids_proj4},
however, neither of them is as close to a \emph{qunatitative} agreement with the experiments 
as the combination of ECC-lipids with ECC-ions. 
In addition, the different response of order parameteres $S^{\alpha _1}$ and $S^{\alpha _2}$ in POPS is also well captured by the model. 
Such a detailed description of the changes of the structural parameters 
demonstrates that including electronic polarization
improves the description of the interaction even for very weakly binding cations like \ce{K^+}. 



\begin{figure}[tbp!] 
  \centering 
  \includegraphics[width=\figwidth]{../img/ecc_popc/density_profiles_ca_cl_wat_phos_models-compar_5-7_NaCl-KCl.pdf}
  \caption{\label{fig:nacl-dens} 
    Number density profiles of \ce{K^{+}}, \ce{Na^{+}} and \ce{Cl^-} along membrane normal axis 
    from the simulations with ECC-lipids and ECC-ions with neutral POPC bilayer.  
    In order to visualize the density profiles with a scale comparable to the profile of \ce{Ca^{2+}} in Fig.~\ref{fig:cacl-dens},  
    the density profiles of~\ce{Cl^-}, \ce{K^+} and \ce{Na^+} ions are divided by 2, and 
    the density profiles of phosphate groups and water are divided by 5 and 200, respectively.  
    The simulation with \ce{NaCl} has $C_{ion}'$=1000~mM, 
    the simulation with \ce{KCl}  has $C_{ion}'$=1100~mM. 
    } 
\end{figure} 


\begin{figure}[tbp!] 
  \centering 
  \includegraphics[width=\figwidth]{../img/ecc_pops/density_profiles_na_k_cl_wat_phos_models-compar_4-6_NaCl-and-KCl-series.pdf}
\todo{Change the labels in the figure so that the $C_{ion}$ tells explicitly \emph{which} ion.}
  \caption{\label{fig:nacl-dens_PCPS} 
    Number density profiles of \ce{K^{+}}, \ce{Na^{+}} and \ce{Cl^-} along membrane normal axis 
    for the negatively charged membrane with the composition of 5\,PC:1\,PS. 
    The top profile shows the simulation without any additional salt concentration, i.e. only with \ce{Na^+} counterions. 
    The middle profile shows the simulation with an additional \ce{KCl} concentration and \ce{Na^+} counterions. 
    The bottom profile shows the simulation with an additional \ce{NaCl} concentration and \ce{Na^+} counterions, which are not distinguished from the added salt. 
    In order to visualize the density profiles with a scale comparable to the profile of \ce{Ca^{2+}} in Fig.~\ref{fig:cacl-dens},  
    the density profiles of~\ce{Cl^-}, \ce{K^+} and \ce{Na^+} ions are divided by 2, and 
    the density profiles of phosphate groups and water are divided by 5 and 200, respectively.  
    } 
\end{figure} 



The difference between the affinity of \ce{Na^+} and \ce{K^+} to neutral and negatively charged membranes
can be described by their relative surface excess with respect to water, $\Gamma ^{w} _{ion}$, 
which is shown in the plots of the density profiles of the ions in Figs.~\ref{fig:nacl-dens} and \ref{fig:nacl-dens_PCPS}. 
Such a quantity compares the adsorption of ions to the adsorption of water molecules at an interface 
without the necessity of defining a Gibbs dividing surface. \citep{chattorajBOOK}
While \ce{K^+} maintains negative value of $\Gamma^{w}_{K}$ even for the negatively charged bilayer,
the value of $\Gamma^{w}_{Na}$ for \ce{Na^+} changes from negative to positive
in a neutral POPC bilayer resp. in a negatively charged bilayer with a composition 5\,PC:1\,PS.
This means that at the given concentration the bilayer interface has a small preference to \ce{Na^+} cations compared to water molecules. 
Interestingly, this value is slightly decreased in the presence of an additional \ce{NaCl} concentration adding also \ce{Cl^-} anions, 
which are not present in the system when only counterions are used
(bottom resp. top plot in Fig.~\ref{fig:nacl-dens_PCPS}. 
The interaction of a neutral POPC bilayer with \ce{NaCl} is discussed in a greater detail in \citep{melcr18}. 

 





 
 


\section{Interactions of neutral and negatively charged phospholipid membranes with \ce{Ca^{2+}} cations}
\label{section:lip-ion_ca}



\begin{figure}[tbp!] 
  \centering 
  \includegraphics[width=\figwidth]{../img/ecc_popc/OrdPars-A-B-PNvec_L14-ECC-lipids_CaCl.pdf}
  \caption{\label{fig:delta_ordPar_CaCl} 
    Changes of the head group order parameters and P-N vector orientation of a POPC bilayer  
    as a function of the CaCl$_2$ concentration in bulk ($C_{ion}$) 
    from simulations at 313 K together with experimental data  
    (DPPC (323\,K) \citep{akutsu81} and POPC (313\,K) \citep{altenbach84}).  
    The error estimate for bulk concentrations is approximately 10\,mM. 
    The order of magnitude larger error in the
    simulation with Lipid14 and ECC-ions is due to unconverged bulk densities  (shown if Fig.~\ref{fig:cacl-dens}) limited by
    the simulation box.  
    Simulation data with Lipid14 and Åqvist ion parameters at 298 K are taken directly from 
    Refs.~\citep{lipid14POPC0mMNaClfiles,lipid14POPC350mMCaClfiles,lipid14POPC350mMCaClfilesNC}. 
  } 
\end{figure} 


\begin{figure}[tbp!] 
  \centering 
  \includegraphics[width=\figwidthsmall]{../img/ecc_pops/order_parameters_changes_ecc-lip_L14_A-B-PN-COO_POPC_cacl.pdf} 
  \includegraphics[width=\figwidthsmall]{../img/ecc_pops/order_parameters_changes_ecc-lip_L14_A-B-PN-COO_POPS_cacl.pdf} 
  \caption{\label{fig:delta_ordPar_CaCl_PCPS} 
    Changes of the head group order parameters $\alpha$, $\beta$ and the orientations of the carboxylate group and the P-N vector  
    of POPC (left) and POPS (right) phospholipids in a POPC:POPS 5:1 bilayer as a function of \ce{CaCl2} concentration 
    in bulk ($C_{ion}$) from simulations with different force fields and experiments at 298 K. \citep{roux90}
    The orientation of the \ce{COO^-} group is defined as 
    the connector from the $\beta$ carbon to the carbon in \ce{COO^-} (stars, bottom right). 
  } 
\end{figure} 



%%% Sum up what is to be presented briefly
The importance of treating polarizability for interactions of phospholipids even with monovalent ions was demonstrated in the previous section. 
Electronic polarization is a non-negligable contribution to the interactions of calcium even in simple aquaeous solutions \ce{CaCl2} \citep{martinek17, kohagen16, Pluharova2014}. 
In this section,
we will show the results from simulations of neutral and negatively charged phospholipid bilayers at varying \ce{CaCl2} concentrations
using the recently developed models ECC-lipids and ECC-ions \citep{melcr18, martinek17}, 
which implicitly include the effects of electronic polarization through electronic continuum correction \citep{leontyev11}. 
Validated with the concept of a lipid electrometer (introduced in section\ref{section:electrometer}),
such implicitly polarizable models yield accurate description of 
the interaction of \ce{Ca^{2+}} with both neutral and negatively charged phospholipids. 

The changes of the head group order parameters $S^\alpha$ and $S^\beta$ from simulations and experiments 
are shown in Fig.~\ref{fig:delta_ordPar_CaCl} for a neutral POPC bialyer, 
and in Fig.~\ref{fig:delta_ordPar_CaCl_PCPS} for a negatively charged bilayer with a compostion 5\,PC:1\,PS. 
For a direct comparison and a connection to the published works by \citet{catte16, nmrlipids_proj4},
results from the simulations with Lipid17 model \citep{lipid17-future} are also included in the figures. 
Although Lipid17 already belongs to the top-performing models in terms of the responses of the head group order parameters in those studies,  
including electronic polarizability to form ECC-lipids improves the results even further.
The effect is probably the most striking for POPS, 
for which also the structure of pure POPS bilayer with only counterions is dramatically improved with the augmentation. 


%%% Discuss the changes of OPs and vectors from neutral and neg. membranes
While the changes induced by \ce{CaCl2} are \emph{qualitatively} correct for Lipid14/17, 
we achieve a \emph{quantitative} agreement with experiments using ECC-lipids model. 
Increasing concentrations of \ce{CaCl2} induce a systematic decrease of the order parameters $S^\alpha$ and $S^\beta$ in POPC 
Although the total magnitude of the response of the order parameteres is comparable in the neutral and the negatively charged bilayers, 
the shape of the changes in the latter shows a steeper onset at low concentrations. 
This is apparently due to the presence of POPS, 
which has a higher affinity to \ce{Ca^{2+}} compared to POPC. 


%%% Present the density profiles (different concentrations for both POPC and mixed)
The increase in the amount of bound calcium cations from pure POPC to the mixed negatively charged bilayer containing POPS
is well demonstrated using the relative surface excess, $\Gamma ^w _{Ca}$,
summarized in Table~\ref{tab:binding}. 
Distributions of \ce{Ca^{2+}}, \ce{Na^+} counterions and also \ce{Cl^-} 
are plotted in Fig.~\ref{fig:cacl-dens} for the neutral POPC bilayer, 
and in Fig.~\ref{fig:cacl-dens_PCPS} for the negatively charged bilayer. 
In contrast to \ce{KCl} or added concentrations of \ce{NaCl}, 
\ce{Na^+} counterions are substituted with \ce{Ca^{2+}} even at low concentrations of calcium. 
The increasing concentration of \ce{CaCl2} and, hence, a higher amount of bound \ce{Ca^{2+}}
also attracts \ce{Cl^-} anions to the bilayer 
as can be seen from its growing density at the interface. 


%%% discuss the table with Gammas and follow up with molecular details, where do cations bind? Support with the spatial density figure
The density profiles of the ions suggest that
the dominant contribution to the binding of \ce{Ca^{2+}} to phospholipid bilayers
comes from the interactions with the phosphate groups in both POPC and POPS. 
This is further validated with a more detailed analysis of the moieties,
which form contacts with the cations, 
which was done by counting contacts between the cations and the oxygen atoms of the lipids
similarly as was done in \cite{melcr18}. 
The threshold for counting a contact was set to $0.3 \mathrm{nm}$, 
which encompasses the first peak of radial distribution function between the cations and the oxygen atoms of the lipids. 

The percentages of the populations of membrane-bound calcium cations for various membrane moieties 
are summarized in Table~\ref{tab:Ca_binding_PCPS}.
Although the lipid ratio in the negatively charged membrane is 5~PC:1~PS,
Even though the negatively charged membrane contains only $18\%$ of POPS, 
approximately half of the total population of bound calcium cations is in contact with PS lipids
with 7\% bound only to them. 
This corroborates the intrinsincally higher affinity of PS lipids to calcium cations compared to PC lipids. 

Relative probabilites of \ce{Ca^{2+}} complexes with a certain number of lipids are shown in Fig.~\ref{fig:cacl_complexes}. 
Calcium cations that are bound only to PC in the mixed bilayer with PS 
behave similarly as in the pure PC bilayer
maintaining similar probabilities for clustering one, two or even three PC lipids together. 
In contrast, PS lipids prefer 1:1 ratio with \ce{Ca^{2+}},
which may also be due to their low molar fraction in the the mixed bilayer. 
In total, however, the negatively charged membrane has its stoichiometry shifted towards complexes of three phospholipids to one calcium. 
This is also reflected in the increased probability of a single phospholipid interacting with two \ce{Ca^{2+}} cations, 
wchich is almost negligable for the neutral POPC bilayer. 


\todo{Continue editing here.}


%The population analysis also suggests 
%that the calcium cations prefer to reside in the phosphate region of the membrane. 
%Such a finding is further validated with a more detailed analysis using Markov state modeling (MSM) \citep{Pande_MSM_paper} \todoi{Add papers reviewing MSM, e.g. recent Pande's paper.}. 
%The set of states of a calcium cation 
%encompassed all possible combinations of up to three surrounding lipids. 
%In the case of PC, we used only the phosphate moitety, which forms the dominant contribution for calcium binding.
%For PS we also distinguished configurations in which calcium interacts with the carboxylate moiety. 
%All possible combinations of such states were used to build a Markov model at a lag time $25\,\mathrm{ns}$ using pyEMMA code by \citet{pyemma} \todoi{add citation for pyemma}. 
%The resulting MSM was validated using Chapman-Kolmogorov test \citep{FrankNoe_papers_MSM} \todoi{Add citation for Noe's papers on MSM, especially Chapman Kolmogorov test.}. 
%The stationary distribution of the states reveals a strong preference of the calcium cations to reside in the phosphate region of the mixed PC-PS membrane. 
%When interacting with PS, configurations with the phosphate moiety from either PC or PS dominate the population.
%Moreover, states containing interactions with the carboxylate group in PS 
%bear higher probability when interacting also with the phosphate group of the same lipid or other lipids. 
This is also reflected as a shift of the mean orientation of the \ce{COO^-} group from $62^\circ$ to $73^\circ$ (420~mM \ce{CaCl2})
measured as the connector of the carbon atoms, which form the bond between the group and the $\beta$-carbon of the phospholipid. 
The interactions of the carboxylate group in PS with calcium and other phosphate groups
sheds light into the qualitatively different response of the head group order parameters $\alpha$ and $\beta$ in PS compared to PC 
(see Figs.~\ref{fig:delta_ordPar_CaCl} and~\ref{fig:delta_ordPar_CaCl_PCPS}). 



The increased response of head group order parameters $\alpha$ and $\beta$ of PC, which form the lipid electrometer concept,
in mixed 5~PC:1~PS bilayer compared to pure PC
is due to higer affinity of the membrane mediated by even a relatively small fraction (1/6) of PS lipids. 
This is in line with the steep onset of the response of the PS head group order parameters at lower concentrations.
The complex response of the head group order parameters of PS lipids 
is due to the confomational change of the carboxylate group that is attracted more towards the phosphate region. 


Also the changes of the P-N vector angle are too pronounced for the Lipid14 model, 
for which the largest tilting toward water phase induced by a $780\,\mathrm{mM}$ 
CaCl$_2$ concentration is approximately 17$^{\circ}$. The corresponding value 
for the ECC-POPC simulation is only 6$^{\circ}$ ($820\,\mathrm{mM}$ CaCl$_2$).  

Within the Lipid14 model, the overestimated changes in the lipid headgroup order parameter of POPC  as functions of the CaCl$_2$ concentration arise both from the overestimated binding affinity and the excessive sensitivity of the headgroup tilt to the bound positive charge. It is plausible to assume that the same applies to the other lipid models tested in a previous study~\citep{catte16}, which underlines the importance of validation of the lipid headgroup order parameter response to the bound charge.  







%%% timings



 
Binding affinities of Ca$^{2+}$ ions to a POPC bilayer in different simulation models were quantified by calculating the relative surface excess of calcium with respect to water molecules, $\Gamma_{\rm ion}^{\rm water}$, from Eq.~\ref{surfexcess}. 
The values of $\Gamma_{\rm ion}^{\rm water}$ 
from different simulations with the same molar concetration of cations with respect 
to water ($C_{ion}'$=350mM) are shown in Table~\ref{tab:binding}. 
As expected from the changes of the lipid headgroup order parameters in Fig.~ \ref{fig:delta_ordPar_CaCl}, the relative surface excess of calcium, $\Gamma_{\rm Ca}^{\rm water}$ = 0.06~nm$^{-2}$, is significantly smaller for the ECC-POPC model than for the other models, 0.13--0.35~nm$^{-2}$. 
 


\begin{figure}[htbp!] 
  \centering 
  \includegraphics[width=\figwidth]{../img/ecc_popc/density_profiles_ca_cl_wat_phos_models-compar_1-4.pdf} 
  \caption{\label{fig:cacl-dens} 
    Number density profiles of \ce{Ca^{2+}}, \ce{Na^{+}} and \ce{Cl^-} along membrane normal starting at the centre of the bilayer 
    for different force fields. 
    In order to visualize the density profiles with a scale comparable to the profile of \ce{Ca^{2+}},  
    the density profiles of~\ce{Cl^-} ions are divided by 2, and 
    the density profiles of phosphate groups and water are divided by 5 and 200, respectively.  
    All simulations with \ce{CaCl2} shown here have the same molar concentration of ions in water ($C_{ion}'$=350~mM). 
    } 
\end{figure} 


\begin{figure}[htbp!] 
  \centering 
  \includegraphics[width=\figwidth]{../img/ecc_pops/density_profiles_ca_na_cl_wat_phos_models-compar_1-3_CaCl2-series.pdf}
\todo{Change $C_{ion}$ to explicitly tell, which cation (\ce{Ca^{2+}}).}
  \caption{\label{fig:cacl-dens_PCPS} 
    Number density profiles of \ce{Ca^{2+}}, \ce{Na^{+}} and \ce{Cl^-} along membrane normal starting at the centre of the bilayer 
    for the negatively charged membrane of a composition 5\,PC:1\,PS
    at various bulk concentrations of \ce{CaCl2} from simulations. 
    All profiles contain \ce{Na^+} counterions and an additional concentration of \ce{CaCl2}. 
    In order to visualize the density profiles with a scale comparable to the profile of \ce{Ca^{2+}},  
    the density profiles of~\ce{Cl^-} ions are divided by 2, and 
    the density profiles of phosphate groups and water are divided by 5 and 200, respectively.  
    } 
\end{figure} 
 


\begin{table}[tb!] 
\centering
  \caption{Bulk concentrations, $C _{ion}$, and molar fractions, $C' _{ion}$, of Ca$^{2+}$;
           relative surface excess of calcium with respect to water ($\Gamma_{Ca}^{\rm water}$); 
           and percentages of the population 
           of bound Ca$^{2+}$ to various moieties 
           in a neutral membrane composed of POPC
           and in a negatively charged membrane with a compostion 5\,PC:1\,PS.
           \label{tab:binding}} 
  \begin{tabular}{ l | c c } 
	                     &  5\,POPC:1\,POPS &  POPC   \\
	\hline
	$C _{ion}\,/\,\mathrm{mM}$  &  $240\pm 10 $  &  $280\pm 10 $  \\
	$C'_{ion}\,/\,\mathrm{mM}$  &  $400\pm 10 $  &  $350\pm 10 $  \\
	$\Gamma_{Ca}^{\rm water}\, / \,\mathrm{nm}^{-2}$  &  $0.24 \pm 0.01 $  &  $0.06 \pm 0.01 $  \\
	\hline
                             &  \multicolumn{2}{c}{ } \\
        interacting moiety   &  \multicolumn{2}{c}{percentage of bound \ce{Ca^{2+}} } \\
	\hline
	     PC              &   57   &  100   \\
	     PO$_4$    in PC &   40   &   67   \\
	     carbonyls in PC &   ~1   &   ~1   \\
	\hline
	     PS              &    7   &        \\ 
	     PO$_4$  in PS   &    2   &        \\
	     COO$^-$ in PS   &    4   &        \\
	     carbonyls in PS &   ~1   &        \\
	\hline
	both PC and PS       &   36   &        \\
  \end{tabular} 
\end{table} 



\begin{figure}[tb!] 
  \centering 
  \includegraphics[width=\figwidth]{../img/ecc_popc/isocontours_r37_ca_O-carb.png} 
  \caption{\label{fig:volmaps} 
    Isocontours of spatial number density of \ce{Ca^{2+}} (dark blue, 0.001~Å$^{-3}$) 
    and POPC carbonyl oxygen atoms (light semi-transparent red, 0.008~Å$^{-3}$, all POPC lipids contribute). 
    Calcium cations localize mostly around phosphate oxygens (oxygens red, phosphorus bronze).
    Interactions with carbonyl oxygens is less likely than with phosphate oxygens, 
    and it is contributed more by other neighbouring phospholipids than by the same lipid. 
    Transparent structures are shown to depict the variability of choline configurations 
    (colour warps from red to blue along the simulation time). 
    The number density was evaluated for each lipid, 
    after its structural alignment using only phosphate group.
    MDAnalysis \citep{mdanalysis2011} library was used for 
    the calculations of the structural alignment and the spatial number density. 
    VMD \citep{hump96} was used for visualisation. 
    Carbon atoms are depicted in cyan, hydrogen atoms in white, oxygen atoms in red, nitrogen in blue.
  } 
\end{figure} 
 
 
\begin{figure}[tb!] 
  \centering 
  \includegraphics[width=\figwidth]{../img/stoichiometry_CaCl2_comparison_Ecc-lipids_PC-vs-PCPS.pdf} \\ 
  \caption{\label{fig:cacl_complexes} 
      Relative probabilities of existence of \ce{Ca^{2+}} complexes 
      with a certain number of lipids.  
      All lipids were taken into account with the exception of the complexes in light green, 
      for which we counted only contacts with POPC from the mixed 5\,PC:1\,PS negatively charged bilayer 
      and calculated the probabilities of the calcium-lipid complexes also only per POPC. 
      Probabilities were taken from simulations with comparable bulk concentrations of calcium around 250~mM. 
  } 
\end{figure} 
 



\subsection{Molecular interaction and binding affinities of Ca$^{2+}$  cations to the mixed POPC:POPS (5:1) membrane} 


%\todo{Stationary distribution: Make a figure documenting the populations of bound \ce{Ca^{2+}} cations (like I have in the presentation) that would accompany Table \ref{tab:Ca_binding_PCPS}. 
%This will roughly correspond to the PC stoichiometry plot \ref{fig:cacl_complexes}. }




\begin{figure}[tb!]
  \centering
  \hfill
\subfloat[neutral PC bilayer]{
  \includegraphics[width=\figwidthsmall]{../img/ecc_popc/histogram_bound_times_ECC-lipids_346mM_CaCl.pdf} 
}\hfill
\subfloat[negatively charged 5\,PC:1\,PS bilayer]{
  \includegraphics[width=\figwidthsmall]{../img/ecc_pops/histogram_bound_times_26CaCl2.pdf}
}\hfill
  \hfill
\todo{Put the scale of the x-axis the same -- or even better -- combine the plots into one!}
  \caption{\label{fig:hist_residence_times}
   Histograms of residence times of \ce{Ca^{2+}} 
   in a neutral membrane composed of POPC (left)
   and in a negatively charged membrane with a compostion 5\,PC:1\,PS (right)
   from simulations with ECC-lipids and ECC-ions.
   The simulation with the neutral membrane has a bulk concentration of calcium $C_{ion} = 280\mathrm{mM}$, 
   the simulation with the negatively charged membrane has a bulk concentration of calcium $C_{ion} = 240\mathrm{mM}$. 
   In the simulation with the neutral membrane, 
   90\% of the residence times of calcium cations are
   shorter than $60\,\mathrm{ns}$, % exactly $53\,\mathrm{ns}$                                                                          
   with the longest observed residence time being $141\,\mathrm{ns}$. 
   In the simulation with the negatively charged membrane, 
   90\% of the residence times of calcium cations are
   shorter than $180\,\mathrm{ns}$, % exactly $53\,\mathrm{ns}$                                                                          
   with the longest observed residence time being $440\,\mathrm{ns}$. 
   }
\end{figure}


Timescales associated with the binding of calcium cation from solution to the membrane
are plotted for each binding event as a histogram in Fig.~\ref{fig:hist_residence_times}. 
Using these plots, we can estimate the upper bound for the residence time of a calcium cation 
to be lower than $60\,\mathrm{ns}$ for pure POPC neutral bilayer 
and shorter than $180\,\mathrm{ns}$ for the mixed 5\,PC:1\,PS negatively charged bilayer. 
The longest observed residence times in the simulations were $141\,\mathrm{ns}$ for the neutral membrane 
and $440\,\mathrm{ns}$ for the negatively charged membrane. 
Both estimates of the residence times come from simulations with comparable concentrations around $250\mathrm{mM}$;
the simulation with the neutral membrane has a bulk concentration of calcium $C_{ion} = 280\mathrm{mM}$, 
whereas the simulation with the negatively charged membrane has a bulk concentration of calcium $C_{ion} = 240\mathrm{mM}$. 

%In addition to such estimates of the time scales, we used the Markov model on top of the simulation with the negatviely charged mixed bilayer
%to calculate the time of the mean first passage of calcium from solution to the membrane (and in reverse) resulting in $55\,\mathrm{ns}$  ($165\,\mathrm{ns}$). 

%\todo{Fluxes: committor analysis, dominant fluxes (table and figure)}
%From the spectrum of the transition matrix, we also observe that 
%the slowest transitions are associated with the binding to two or three phosphate moieties from either PC or PS,
%which are also among the states with the highest probabilities.
%The net fluxes of calcium cations from solution to such states also form a large contribution to the total flux ($\approx 45\%$). 
%\todo{Make a table and a figure of the state probabilities and fluxes to support this statement.}. 
%Analysis of the possible binding pathways of calcium cations to the negatively charged mixed bilayer 
%reveal that the cations mostly enter the membrane bound states directly from solution 
%without complicated transitions at the time scales of the order of the lag time of the Markov state model, $25\,\mathrm{ns}$. 










 


 
\subsection{Molecular interactions between Ca$^{2+}$ cations and POPC oxygens} 
We analyzed the ratio of the number of calcium cations bound to either phosphate or carbonyl moieties and the total number of bound cations in our POPC bilayers as done previously in Ref.~\citep{javanainen17}. A maximum distance of 0.3~nm from any lipid oxygen is used to define a bound calcium. The results from ECC-POPC simulation in Table~\ref{tab:binding} show that almost all (99\%) of the bound Ca$^{2+}$ ions are in direct contact with phosphate oxygens. From these ions, only one third (32\%) also interacts with the carbonyl oxygens, while the interaction of calcium ions with carbonyl oxygens only is rare (1\%). The most abundand interaction scenarios between Ca$^{2+}$ ions and phosphate oxygens are visualized using the probability density isocontours in Fig.~\ref{fig:volmaps}. While higher concentrations of \ce{CaCl2} increase the number of contacts per lipid, the distribution of contacts between phosphate and carbonyl oxygens is not affected. 
 
\todo{Analyze the orientation of the carbonyls and plot it as a violin plot/probability density. Change the following discussion afterwards.}
In the case of POPC, the $C_2$ segment in {\it sn}-2 chain shows a low order parameter with a small forking as measured in experiments by \citet{seelig75,schindler75,gawrisch92}. 
This feature has been suggested to indicate that the carbonyl
of {\it sn}-2 chain is directed towards the water phase, in contrast to the
carbonyl in {\it sn}-1 chain, which would orient more along the bilayer
plane~\cite{seelig75,schindler75,gawrisch92}. This may be an important
feature for the ion binding details, which it is not fully reproduced by other
available lipid models~\cite{ollila16}.

 
In conclusion, the results suggest that calcium ions bind specifically to the phosphate oxygens, occasionally interacting also with the carbonyls of the PC lipids. This is in a qualitative agreement with previous conclusions from several experimental studies~\citep{hauser76, hauser78, herbette84, cevc90, binder02}. However, the present results suggest, in 
agreement with experiments, an overally weaker binding to the bilayer, in particular with a lower relative binding affinity to the carbonyls than inferred from previous MD simulation studies~\citep{bockmann03, bockmann04, melcrova16, javanainen17}. 

 
\subsection{Binding stoichiometry of \ce{Ca^{2+}} cations to POPC membrane} 
Simple binding models have been used previously to interpret the same experimental data \citep{altenbach84,macdonald87} as employed in this work to validate the simulation models (Fig.~\ref{fig:delta_ordPar_CaCl}). In particular, NMR data concerning the PC headgroup order parameters response and atomic absorption spectra were explained best using a ternary complex binding model with a binding stoichiometry of one \ce{Ca^{2+}} per two POPC lipids~\citep{altenbach84}. Nevertheless, a Langmuir adsorption model assuming a \ce{Ca^{2+}}:POPC stoichiometry of 1:1 also provided a good fit to the experimental data when considering \ce{CaCl2} at low concentrations only~\citep{macdonald87}. 
 
 
In this work, we reproduce the same experimental data used to infer binding stoichiometries employing our ECC-POPC model. Thanks to our simulations, we have a direct access to atomistic details of the binding stoichiometry without a need for any binding model as employed for interpreting in experiments~\citep{altenbach84, macdonald87}.
To evaluate the relative propensities for each of the stoichiometric complexes (i.e.,~1~Ca$^{2+}$:~n~POPC),
we calculated for each bound Ca$^{2+}$ the number of POPC molecules having oxygen atoms within a distance of 0.3~nm.
Results from the POPC bilayer simulation with a 285~mM bulk concentration of CaCl$_2$ are shown in Fig.~\ref{fig:cacl_complexes}. 
We found the largest propensity for the 1:2 complex (41\%), with probabilities of complexes with the stoichiometries of 1:1~(25\%)~and 1:3~(34\%)~being only slightly lower. This suggests a more complex binding model than considered in a simple 1:2 ternary complex model previously. Nevertheless, with a broad brushstroke, the simulation data can be viewed such that one calcium binds to two lipids on average, because the probabilities of the complexes with 1 or 3 lipids are almost equal to each other  (and complexes with more than three lipids per one calcium ion were not observed). This probably explains why the simple the ternary complex model fits adequately the experimental data, as well as the ECC-POPC simulation results (see Fig.~S3 in SI). 
 
 
 
\subsection{Residence times of \ce{Ca^{2+}} cations in the POPC membrane} 
 
Equilibration of \ce{Ca^{2+}} ions at a POPC bilayer in MD simulations is a microsecond time scale process with current force fields, such  as CHARMM36 and Slipids force fields~\citep{javanainen17}. This suggests that at least several microseconds are required to reach the ion binding/unbinding equilibrium. 
To quantify the exchange of ions between the membrane and aqueous solution in simulations, we evaluated residence times of ions bound to the membrane. Within our analysis, an ion is considered to be bound when it is within 0.3~nm from any oxygen atom belonging to a POPC molecule. 
 
The histograms of residence times of \ce{Ca^{2+}} in a POPC bilayer ($C_{ion}'$ = 450~mM) from simulations with  
ECC-POPC and CHARMM36 (simulation from Refs.~\citep{javanainen17,zenodo.259376}) are shown in Fig.~S4 in SI. 
In the CHARMM36 simulation, a significant number of the calcium ions is bound to the membrane for the whole length of the trajectory (800~ns). 
In contrast, at least an order of magnitude faster bound/unbound calcium exchange is observed within the ECC-POPC model, 
where 90\% of the \ce{Ca^{2+}} residence times to a POPC membrane are shorter than $60\,\mathrm{ns}$. The longest observed 
residence time is around 150~ns, which is below the total length of the simulation used for analysis, i.e., 200~ns. 
Note that these results are in line with the experimental estimate that the residence time of \ce{Ca^{2+}} at each PC 
headgroup is of the order of $10\,\mu\mathrm{s}$~\citep{altenbach84}. 
 
In summary, the results from the ECC-POPC model suggest that the exchange of calcium between the POPC bilayer and the solvent occurs within the $\sim$100~ns timeframe, which is significantly faster than observed in simulations emloying most of the presently available lipid models~\citep{javanainen17}. Sodium cations exhibit an even more rapid exchange between the membrane and the aqueous solution. Our results suggest that simulations with a length of several hundreds of nanoseconds are sufficient to simulate alkali and alkali earth ion binding to phospholipid bilayers in equilibrium when realistic force fields are used. This has not been the case with previous lipid force fields, which overestimate the binding strength of the sodium and, in particular,  calcium cations \citep{javanainen17, catte16}. 
 
 









