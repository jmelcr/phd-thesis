\documentclass[12pt,a4paper]{report}

\usepackage{mhchem} 
\usepackage[a-2u]{pdfx}

%% Character encoding: usually latin2, cp1250 or utf8:
\usepackage[utf8]{inputenc}

%% Prefer Latin Modern fonts
\usepackage{lmodern}


\begin{document}

\chapter*{Simulation of processes in cellular membranes}

\section*{Abstract}

Many important processes in cells involve ions, e.g., 
fusion of synaptic vesicles with neuronal cell membranes 
is controlled by a divalent cation \ce{Ca^{2+}};
and the exchange of \ce{Na^+} and \ce{K^+}
drives the the fast electrical signal transmission in neurons. 
We have investigated model phospholipid membranes and their interactions 
with these biologically relevant ions. 
Using state-of-the-art molecular dynamics simulations,
we accurately quantified their respective affinites 
towards neutral and negatively charged phospholipid bilayers. 
In order to achieve that,
we developed a new model of phospholipids termed ECC-lipids,
which accounts for the electronic polarization
via the electronic continuum correction implemented as charge rescaling. 
Our simulations with this new force field 
reach for the first time a quantitative agreement 
with the experimental lipid electrometer concept 
for POPC as well as for POPS with all the studied cations. 
We have also examined the effects of transmembrane voltage on phospholipid bilayers. 
The electric field induced by the voltage 
exists exclusively in the hydrophobic region of the membrane,
where it has an almost constant strength. 
This field affects the structure of nearby water molecules 
highlighting its importance in electroporation. 

\end{document}
