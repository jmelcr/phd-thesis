\documentclass[12pt,a4paper]{report}

\usepackage{mhchem} 
\usepackage[a-2u]{pdfx}

%% Character encoding: usually latin2, cp1250 or utf8:
\usepackage[utf8]{inputenc}

%% Prefer Latin Modern fonts
\usepackage{lmodern}


\begin{document}

\chapter*{Simulace of procesů v buněčných membránách}

\section*{Abstrakt}

Mnoho důležitých procesů v buňkách probíhá prostřednictvím iontů.
Například fúze synaptických váčků s membránami nervových buněk 
je kontrolována dvojmocným kationtem \ce{Ca^{2+}},
zatímco výměna \ce{Na^+} a \ce{K^+}
řídí rychlý elektrický přenos vzruchů neurony.  
Vyšetřili jsme modelov{\'e} fosfolipidov{\'e} membrány a jejich interakce
s těmito biologicky relevantními ionty. 
S použitím molekulárně dynamických simulací
jsme přesně určili jejich vzájem{\'e} afinity 
vůči neutrálním a negativně nabitým fosfolipidovým dvojvrstvám. 
K tomu bylo nutn{\'e} vyvinout
nov{\'e} vylepšen{\'e} modely fosfolipidů nazvan{\'e} ECC-lipids,
kter{\'e} obsahují polarizaci elektronů 
pomocí korekce na elektronov{\'e} kontinuum implementovan{\'e} přeškálováním nábojů.
Naše simulace s tímto novým silovým polem 
poprv{\'e} dosahují kvantitativní shody 
s experimentálně zjištěným konceptem lipidov{\'e}ho elektrometru
pro POPC a i pro POPS se všemi studovanými kationty. 
Kromě toho jsme tak{\'e} zkoumali vliv transmembránov{\'e}ho napětí na fosfolipidov{\'e} dvojvrstvy.  
Elektrick{\'e} pole indukovan{\'e} napětím se vyskytuje
výhradně v hydrofóbní části membrány, 
kde má t{\'e}měř konstantní intenzitu. 
Toto pole ovlivňuje strukturu blízkých molekul vody, 
která je podstatným faktorem při elektroporaci membrán. 

\end{document}
