\chapter{Biological membranes}

 Introduction to the field. 
 \cite{Student08}

 Phospholipids. 

 Interaction with cations -- significance. 

 What is known in this respect. 

 Contradiction between simulation/experimental studies 
(use what's in the paper PCCP NMRLipids II). 

Experimental studies. 

\section{NMR order parameter measurements}
 
  More general statements about NMR  -- especially the way $S_{CH}$ is measured. 

\subsection{Electrometer concept} \label{section:electrometer} 

Comparing MD simulation to NMR experiments, we can validate the ion 
binding affinity in lipid bilayer simulations using the 'electrometer concept'~ \cite{seelig87, catte16}. 
This method is based on the experimental observation that the C-H bond order parameters of $\alpha$ and $\beta$ carbons in a PC lipid head group (Fig.~\ref{simVSexpNOions}) are proportional to the amount of charge bound per lipid~\cite{seelig87}. 
The order parameters for all C-H bonds in lipid molecules can be accurately measured using $^2$H NMR or $^{13}$C NMR techniques \cite{ollila16}. 
From MD simulations the order parameters can be calculated using the relation 
\begin{equation}\label{OP} 
S_{\rm CH}=\frac{1}{2}\langle 3\cos^2\theta -1 \rangle, 
\end{equation} 
where $\theta$ is the angle between the C-H bond and the membrane normal. 
Angular brackets point to the average over all sampled configurations. 
 
The relation between the amount of the bound charge per lipid, $X^\pm$, and the head group order parameter change, $\Delta S_{\rm{CH}}^{i}$, is empirically quantified as~\cite{seelig87,ferreira16} 
\begin{equation}\label{OPchangeEQ} 
\Delta S_{\rm{CH}}^{i}= S_{\rm{CH}}^{i}(X^\pm)-S_{\rm{CH}}^{i}(0) \approx m_i \frac{4 }{3\chi}X^\pm, 
\end{equation} 
where $i$ refers to either the $\alpha$ or $\beta$ carbons, $S_{\rm{CH}}^{i}(0)$ denotes the order parameter in the absence of bound charge, $\chi$ is the quadrupole coupling constant ($\chi \approx$\,167\,kHz), and $m_i$ is an empirical constant depending on the valency and location of the bound charge. 
 
 
The measured change of the order parameter depends on the head group response to the bound charge and on the amount of the bound charge (\textit{i.e.,} $m_i$ and $X^\pm$ in Eq.~\ref{OPchangeEQ}, respectively).  
The empirical factor $m_i$ has to be adequately quantified before the electrometer concept can be used to analyze the binding affinities. 
This calibration has been done experimentally for a wide range of systems~\cite{seelig87, beschiasvili91}. 
To calibrate the response of the head group order parameters to the bound charge in simulations, we use experimental data for a strong cationic surfactant dihexadecyldimethylammonium bromide  (DHAB) mixed with a POPC bilayer~\cite{scherer89}. DHAB\\[0.5cm] 
%\vspace{0.5cm} 
%\chemfig{ -[:0,3.5,,,draw=none]\chemabove{N}{\scriptstyle\oplus} (-[:150]H_3C)(-[:210]H_3C)(-[:330]{(}CH_2{)}_{15}-CH_3)(-[:30]{(}CH_2{)}_{15}-CH_3) }   
%\vspace{0.5cm} \\ 
is a cationic surfactant having two acyl chains and bearing a unit charge at the hydrophilic end. 
Due to its structure it is expected to locate in the bilayer similarly to the phospholipids and its molar ratio then gives directly the amount of bound unit charge per lipid $X^\pm$ in these systems~\cite{scherer89}. 
 
\section{SAXS? + other techniques (just to make the lit. research complete) -- probably not.}

The structures of lipid bilayers in simulations without ions were validated against NMR by calculating the order parameters for the C-H bonds and against \mbox{X-ray} scattering experiments by evaluating the scattering form factors. 
NMR order parameters validate the structures sampled by the individual lipid molecules with atomic resolution. 
The simulated order parameters were calculated for all C-H bonds in lipid molecules from Eq. \ref{OP}. 
Scattering form factors validate the dimensions of the lipid bilayer (i.e., the bilayer thickness and area per molecule). 
Form factors were calculated using a relation 
 
\begin{equation} 
  F(q) = \int _{-D/2} ^{D/2} \left ( \rho_{el}(z) - \rho_{el}^s \right ) \cos (zq_z) \mathrm{d}z, 
  %F(q) = \int _{-D/2} ^{D/2} \left ( \sum _\alpha f_\alpha (q_z) n_\alpha (z) - \rho _s \right ) \exp (izq_z) \mathrm{d}z, 
\end{equation} 
 
\noindent where $\rho_{el} (z)$ is the total electron density, $\rho_{el}^s$ is the electron density of the solvent far in the aqueous bulk, and $z$ is the distance from the membrane center along its normal with $D/2$ being half of the unit cell size.   

