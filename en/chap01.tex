\chapter{Biological membranes}
\label{chap:intro}

 \textbf{Basically brief recap of preface enriched with extras on membranes.}

 Introduction to the field. 
 Inner/outer environment. 
 Membrane separates and helps maintaining non-equilibrium unequal distributions of cations/electrolytes. 

 Interaction with cations and its significance -- already in preface?

 Exemplify things with the following: \\
 Processes in Neurons. 
 Transmembrane potential. 
 Action potential. 

\section{Membrane components}

\subsection{Phospholipids}

\subsection{Cholesterol}

\subsection{Other lipids: Myelination -- chop into a short enrichment -- a particular composition}
Myelin has two important advantages:\textbf{ fast conduction speed and energy efficiency}. For axons larger than a minimum diameter (roughly 1 micrometre), myelination increases the conduction velocity of an action potential, typically tenfold. Conversely, for a given conduction velocity, myelinated fibers are smaller than their unmyelinated counterparts. For example, action potentials move at roughly the same speed (25 m/s) in a myelinated frog axon and an unmyelinated squid giant axon, but the frog axon has a roughly 30-fold smaller diameter and 1000-fold smaller cross-sectional area. Also, since the ionic currents are confined to the nodes of Ranvier, far fewer ions "leak" across the membrane, saving metabolic energy. This saving is a significant selective advantage, since the human nervous system uses approximately 20\% of the body's metabolic energy.

In nature, myelinated segments are generally long enough for the passively propagated signal to travel for at least two nodes while retaining enough amplitude to fire an action potential at the second or third node. Thus, the safety factor of saltatory conduction is high, allowing transmission to bypass nodes in case of injury. However, action potentials may end prematurely in certain places where the safety factor is low, even in unmyelinated neurons; a common example is the branch point of an axon, where it divides into two axons.

\subsection{Non-lipids: protein and sugar, glycocalyx}

\section{Model membranes, Phospholipid bilayers }

 Model membranes and membrane models in experiments and in simulations.
 Phospholipid bilayers or just bilayers with various composition. 

 Supported bilayers, vesicles, SUV, LUV, GUV, multilamellar vesicles




\section{Interactions of cations with phospholipid bilayers}

 Several chosen Experimental studies. 

 Contradiction between simulation/experimental studies 
 (use what's in the paper PCCP NMRLipids II). 

Due to its high physiological importance --- nerve cell signalling being the prime example ---
interaction of cations with phospholipid membranes
has been widely studied via theory, simulations, and experiments.
The relative ion binding affinities are generally agreed to
follow the Hofmeister series~\cite{eisenberg79,cevc90,tocanne90,binder02,celma07,leontidis09,vacha09a,klasczyk10,harb13}, 
however,
consensus on the quantitative affinities is currently lacking.
Until 1990, the consensus (documented in two extensive reviews~\cite{cevc90,tocanne90}) was that
while  multivalent cations interact significantly with phospholipid bilayers,
for monovalent cations (with the exception of Li$^+$) the interactions are weak.
This conclusion has since been strengthened by further
studies showing that bilayer properties remain unaltered upon the addition of sub-molar concentrations of monovalent 
salt~\cite{binder02,pabst07,filippov09}.
Since 2000, however, another view has emerged, suggesting much stronger interactions between phospholipids and monovalent cations, and strong Na$^{+}$ binding in particular~\cite{bockmann03,bockmann04,vacha09a,manyes05,manyes06,fukuma07,leontidis09,ferber11,morata12,klasczyk10,harb13}.

The pre-2000 view has the experimental support that
(in contrast to the significant effects caused by any multivalent cations)
sub-molar concentrations of NaCl have a negligible effect on
phospholipid infrared spectra~\cite{binder02},
area per molecule~\cite{pabst07},
dipole potential~\cite{clarke99},
lateral diffusion~\cite{filippov09},
and choline head group order parameters~\cite{akutsu81};
in addition, the water sorption isotherm of a NaCl--phospholipid system
is highly similar to that of a  pure NaCl solution
--- indicating that the ion--lipid interaction is very weak~\cite{binder02}. 

The post-2000 'strong binding' view rests on experimental and above all simulational findings.
At sub-molar NaCl concentrations, the rotational and translational dynamics of membrane-embedded fluorescent probes decreased~\cite{bockmann03,vacha09a,harb13}, and atomic force microscopy (AFM) experiments showed changes in bilayer hardness~\cite{manyes05,manyes06,fukuma07,ferber11,morata12};
in atomistic molecular dynamics (MD) simulations, phospholipid bilayers consistently bound Na${^+}$,
although the binding strength depended on the model used~\cite{bockmann03,bockmann04,sachs04,berkowitz06,cordomi08,cordomi09,valley11,berkowitz12}.

Some observables have been interpreted in favour of both views. For example,
as the effect of monovalent ions (except Li$^+$)  on the phase transition temperature is tiny
(compared to the effect of multivalent ions), it was initially interpreted 
as an indication that only multivalent ions and Li$^+$ specifically bind to phospholipid bilayers~\cite{cevc90}; 
however, such a small effect in calorimetric measurements was later interpreted to indicate that also
Na$^+$ binds~\cite{bockmann03,klasczyk10}.
Similarly, the lack of significant positive electrophoretic mobility
of phosphatidylcholine (PC) vesicles in the presence of NaCl
(again in contrast to multivalent ions and Li$^+$)
suggested weak binding of Na$^+$~\cite{eisenberg79,tatulian87,manyes05,manyes06,klasczyk10};
%NaCl increases the (initially negative) zeta potential to only about zero,
%whereas positive zeta potentials are generally reached with
however, these data were also explained by a countering effect of the Cl$^-$ ions~\cite{berkowitz06,knecht13}.
Furthermore, to reduce the area per lipid in scattering experiments, molar concentrations of NaCl were required~\cite{pabst07}, indicating weak ion--lipid interaction;
in MD simulations, however, already orders of magnitude lower concentrations resulted in Na$^+$ binding and a clear reduction of area per lipid~\cite{bockmann03,cordomi08}.
Finally, lipid lateral diffusion was unaltered by NaCl in noninvasive NMR experiments~\cite{filippov09};
%suggesting that the fluorescence results arise from Na$^{+}$ interactions with probes rather than with lipids.
%This is pointed out in Conclusions, which I think is the best place for it. -markus.
however, as it was reduced upon Na$^+$ binding in simulations,
the reduced lateral diffusion of fluorescent probes~\cite{bockmann03,vacha09a,harb13}
has been interpreted to support the post-2000 'strong binding' view.

In this paper, we set out to solve the apparent contradictions
between the pre-2000 and post-2000 views.
To this end, we employ the 'molecular electrometer' concept,
according to which the changes in the C--H order parameters of the $\alpha$ and $\beta$ carbons 
in the phospholipid head group (see Fig.~\ref{POPCstructure}) can be used to measure the ion affinity for a
PC lipid bilayer~\cite{brown77,akutsu81,altenbach84,seelig87,scherer89}.
As the order parameters can be accurately measured in experiments and directly compared to 
simulations~\cite{ollila16}, applying the molecular electrometer as a function of cation concentration allows the 
comparison of binding affinity between simulations and experiments.
In addition to demonstrating the usefulness of this general concept,
we show that the response of the $\alpha$ and $\beta$ order parameters to penetrating cations
is qualitatively correct in MD simulations, but that in several  models the affinity of Na$^{+}$ for PC bilayers
is grossly overestimated.
Moreover, we show that the accuracy of lipid--Ca$^{2+}$ interactions 
in current models is not enough for atomistic resolution interpretation of NMR experiments. 


\section{NMR order parameter measurements}
 
  More general statements about NMR  -- especially the way $S_{CH}$ is measured. 

\subsection{Electrometer concept} \label{section:electrometer} 

Comparing MD simulation to NMR experiments, we can validate the ion 
binding affinity in lipid bilayer simulations using the 'electrometer concept'~ \citep{seelig87, catte16}. 
This method is based on the experimental observation that the C-H bond order parameters of $\alpha$ and $\beta$ carbons in a PC lipid head group (Fig.~\ref{simVSexpNOions}) are proportional to the amount of charge bound per lipid~\citep{seelig87}. 
The order parameters for all C-H bonds in lipid molecules can be accurately measured using $^2$H NMR or $^{13}$C NMR techniques \citep{ollila16}. 
From MD simulations the order parameters can be calculated using the relation 
\begin{equation}\label{OP} 
S_{\rm CH}=\frac{1}{2}\langle 3\cos^2\theta -1 \rangle, 
\end{equation} 
where $\theta$ is the angle between the C-H bond and the membrane normal. 
Angular brackets point to the average over all sampled configurations. 
 
The relation between the amount of the bound charge per lipid, $X^\pm$, and the head group order parameter change, $\Delta S_{\rm{CH}}^{i}$, is empirically quantified as~\citep{seelig87,ferreira16} 
\begin{equation}\label{OPchangeEQ} 
\Delta S_{\rm{CH}}^{i}= S_{\rm{CH}}^{i}(X^\pm)-S_{\rm{CH}}^{i}(0) \approx m_i \frac{4 }{3\chi}X^\pm, 
\end{equation} 
where $i$ refers to either the $\alpha$ or $\beta$ carbons, $S_{\rm{CH}}^{i}(0)$ denotes the order parameter in the absence of bound charge, $\chi$ is the quadrupole coupling constant ($\chi \approx$\,167\,kHz), and $m_i$ is an empirical constant depending on the valency and location of the bound charge. 
 
 
The measured change of the order parameter depends on the head group response to the bound charge and on the amount of the bound charge (\textit{i.e.,} $m_i$ and $X^\pm$ in Eq.~\ref{OPchangeEQ}, respectively).  
The empirical factor $m_i$ has to be adequately quantified before the electrometer concept can be used to analyze the binding affinities. 
This calibration has been done experimentally for a wide range of systems~\citep{seelig87, beschiasvili91}. 
To calibrate the response of the head group order parameters to the bound charge in simulations, we use experimental data for a strong cationic surfactant dihexadecyldimethylammonium bromide  (DHAB) mixed with a POPC bilayer~\citep{scherer89}. DHAB\\[0.5cm] 
%\vspace{0.5cm} 
%\chemfig{ -[:0,3.5,,,draw=none]\chemabove{N}{\scriptstyle\oplus} (-[:150]H_3C)(-[:210]H_3C)(-[:330]{(}CH_2{)}_{15}-CH_3)(-[:30]{(}CH_2{)}_{15}-CH_3) }   
%\vspace{0.5cm} \\ 
is a cationic surfactant having two acyl chains and bearing a unit charge at the hydrophilic end. 
Due to its structure it is expected to locate in the bilayer similarly to the phospholipids and its molar ratio then gives directly the amount of bound unit charge per lipid $X^\pm$ in these systems~\citep{scherer89}. 
 




