\chapter{Biological membranes}
\label{chap:intro}

%A historical review on lipid membrane models \citep{Mouritsen2011}. 
%Lipid rafts references are \citep{Simons2004,Hancock2006,Lingwood2010,Quinn2010,Kraft2013}. 

All living cells separate their inner environment from the exterior by a membrane 
preventing molecules from a free diffusion in or out. 
This allows the cells to maintain specific conditions required for their functioning. 
Cellular membranes are mainly composed of lipids and proteins, 
while enrichment with other molecules, e.g., sugars and saccharides forming glycocalyx, 
is often required for their specific function.  \citep{reitsma07} 
In such a dynamically structured fluid mosaic,  % dynamically structured mosaic model
lipids self-assemble into a lipid bilayer \citep{vereb03,Mouritsen2011,Vattulainen2011}.

Biological membranes are systems with a very complex structure. 
It is, hence, convenient to use simplified models of such membranes, 
which allow us to focus on the basic principles and properties. 
Lipid bilayers are often used as such model membranes 
being accessbile for both experimental and simulation studies. 
In particular, the zwitterionic phosphatidylcholine (PC) bilayers are
used to study the role of ions in complex biological
systems~\citep{scherer87,seelig90,cevc90, vacha09a, javanainen17}.
In experiments, there are several commonly used model membrane systems --
liposomes, 
micelles, 
unilamellar vesicles of varying sizes from giant to small, 
supported bilayers,
bicelles,
and nanodiscs. \citep{keller2017, MolBiolCell, marsh13}
Despite their simplicity compared to cellular membranes,
even such model membranes pose significant challenge for science today. 
\citep{pohl18, melcrova16, javanainen17, magarkar2017, botan15, catte16, Kulig14b, Kulig14, Pluhackova2016, vacha09a}






\section{Interactions of cations with phospholipid bilayers}

Interactions of cations with cellular membranes play a key role in many biological processes. 
The binding of Na$^+$ and Ca$^{2+}$ to PC bilayers has been widely studied both in
experiments~\citep{akutsu81,altenbach84,seelig90,cevc90,tocanne90,binder02,pabst07,uhrikova08}
and simulations~\citep{magarkar2017, bockmann03,bockmann04,berkowitz12,melcrova16,javanainen17, catte16, nmrlipids_proj4}.
The details of ion binding are, however, not fully consistent in the literature.
The relative binding affinities of ions are generally agreed to
follow the Hofmeister series~\citep{eisenberg79,cevc90,tocanne90,binder02,celma07,leontidis09,vacha09a,klasczyk10,harb13}, 
however,
there is no consensus on the quantitative aspects of the ion affinities. 

Interpretations of spectroscopic methods, like nuclear magnetic resonance (NMR), 
x-ray scattering, and infrared spectroscopy suggest that monovalent alkali cations (with the exception of Li$^+$) 
exhibit negligible binding to PC lipid bilayers, 
while multivalent cations interact more strongly 
\citep{cevc90,tocanne90, hauser76,hauser78,herbette84,altenbach84,uhrikova08}.
In particular, submolar concentrations of NaCl have a negligible effect on
phospholipid infrared spectra~\citep{binder02},
area per molecule~\citep{pabst07},
dipole potential~\citep{clarke99},
lateral diffusion~\citep{filippov09},
and choline head group order parameters~\citep{akutsu81};

Another view, emerging more rencently, suggests much stronger interactions between phospholipids and monovalent cations. 
For instance, submolar concentrations of \ce{NaCl} reduce rotational and translational dynamics of membrane-embedded fluorescent probes \citep{bockmann03,vacha09a,harb13}, 
and changes in the hardness of bilayers are also reported from atomic force microscopy experiments \citep{manyes05,manyes06,fukuma07,ferber11,morata12}.
In addition, atomistic molecular dynamics simulations consistently predict strong interactions of Na${^+}$ with phospholipids bilayers,
which, however, depend on the employed model \citep{bockmann03,bockmann04,sachs04,berkowitz06,cordomi08,cordomi09,valley11,berkowitz12,catte16,nmrlipids_proj4,melcr18}.

In our recent work~\citep{catte16}, we addressed the apparent controversies 
using the molecular electrometer concept, 
where the experimentally (by NMR) measured changes of certain head group order parameters 
are related to the amount of charge bound to a bilayer. \citep{brown77,akutsu81,altenbach84,seelig87,scherer89}.
This concept is introduced and further discussed in the following section~\ref{section:electrometer}. 
We have shown that the response of the PC head group order parameters $\alpha$ and $\beta$ 
to cations bound to the bilayer is qualitatively correct in MD simulations, 
but several models grossly overestimate the affinity of \ce{Na^+} to such bilayers. 
Moreover, we find that the interactions of \ce{Ca^{2+}} cations with PC bilayers
in current simulation models are not accurate enough for interpreting NMR experiments. 
We argued that the lack of electronic polarizability in the simulations 
can be responsible for the disagreement. 
Consequently, we developed a model of POPC, 
which accounts for electronic polarization through electrnoic continuum correction (see section~\ref{section:ecc}), 
and which yields accurate response of the head group order parameteres to various monovalent and divalent cations. \citep{melcr18}
This model was also extended to other lipids, which are introduced in the section~\ref{section:ecc-lipids}, 
and it was employed in Chapter~\ref{chap:results} 
providing a simulation-based supporting evidence to the older view,
which suggests that \ce{Na^+} and \ce{K^+} interact weakly, 
while \ce{Ca^{2+}} specifically binds to phospholipid bilayers.







% &=& &=& &=& &=& &=& &=& &=& &=& &=& &=& &=& &=& &=& &=& 
% &=&     SEPARATOR   &=& &=& &=& &=& &=& &=& &=& &=& &=& 
% &=& &=& &=& &=& &=& &=& &=& &=& &=& &=& &=& &=& &=& &=& 


\section{NMR molecular electrometer concept} \label{section:electrometer} 

Ion binding in lipid bilayers can be experimentally quantified
by measuring the changes of the head group order parameters in lipids.
In the case of the head group order parameters $\alpha$ and $\beta$ in phosphatidylcholine
(see Fig.~\ref{fig:simVSexpNOions} for the definition of the order parameters)
such changes are known under the term ''electrometer concept'' \citep{seelig87,catte16, ollila16}. 
It is experimentally observed that the C-H bond
order parameters of $\alpha$ and $\beta$ carbons in a PC lipid head group
are proportional to the amount of charge, positive or negative, bound per lipid~\citep{seelig87}.
The change of the order parameters measured with varying concentrations of aqueous ions 
can be then related to the amount of bound ions.

The order parameters can be robustly and accurately determined from both MD simulations and  NMR experiments. 
Hence, the electrometer concept can be used to compare the two techniques
in terms of the ion binding affinities to lipid bilayers containing PC phospholipids.  \citep{catte16,ollila16} 

From MD simulations the order parameters can be calculated using the definition
\begin{equation}\label{OP} 
S_{\rm CH}=\frac{1}{2}\langle 3\cos^2\theta -1 \rangle, 
\end{equation} 
where $\theta$ is the angle between the \ce{C-H} bond and membrane
normal and the average, denoted with pointed brackets, is taken over all sampled configurations \citep{ollila16}.

The order parameters for all C-H bonds in lipid molecules, including
$\alpha$ and $\beta$ segments in head group, can be accurately measured
using $^2$H or $^{13}$C NMR techniques \citep{ollila16}. 
The relation between the amount of the bound charge per lipid,  $X^\pm$, and
the head group order parameter change, $\Delta S_{\rm{CH}}^{i}$,
is empirically quantified as~\citep{seelig87,ferreira16}

\begin{equation}\label{OPchangeEQ} 
\Delta S_{\rm{CH}}^{i}= S_{\rm{CH}}^{i}(X^\pm)-S_{\rm{CH}}^{i}(0) \approx m_i \frac{4 }{3\chi}X^\pm, 
\end{equation} 
where $S_{\rm{CH}}^{i}(0)$ denotes the order parameter in the absence of any bound charge,
$i$ refers to either $\alpha$ or $\beta$ carbon,
$m_i$ is an empirical constant depending on the valency and the position of the bound charge,
and $\chi \approx$\,167\,kHz is the quadrupole coupling constant, 
with the experimental value for POPC from the works by \citet{seelig77,Davis83}.

The measured change of the order parameter depends on the head group response to the bound charge 
and on the amount of the bound charge (\textit{i.e.,} $m_i$ and $X^\pm$ in Eq.~\ref{OPchangeEQ}, respectively).  
The quantification of the empirical factor $m_i$ has been done experimentally for a wide range of systems
to demonstrate the robustness of the electrometer concept and its negligable chemical specificity~\citep{seelig87, beschiasvili91}. 
The measurements of the electrometer response to cationic surfactants 
can be exploited in simulations to compare the response of the head group order parameters 
with a known amount of bound surface charge per lipid,
i.e., $X^\pm$ in Eq.~\ref{OPchangeEQ},
as all molecules of the surfactant partition into the bilayer. 


